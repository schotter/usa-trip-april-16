Der Urlaub neigt sich so langsam dem Ende, was mir ganz recht ist, denn seit ein paar Tagen ist meine Kreditkarte gesperrt.
Daheim erfahre ich dann von meiner Bank, dass ich
\begin{quote}
	am 27.04.2016 um 16:59:05 bei Rajeshwari Electricals in \TOWN{Bangalore}, also Indien, 
\end{quote}
Elektrokram gekauft haben will.
Ohne den Christian wäre ich jetzt ziemlich aufgeschmissen.
Wie kann man auch nur mit einer Kreditkarte los ziehen\dots

Zum Einkaufen sind wir drei Meilen zur Mall gelaufen und wieder zurück.
Dann war es auch schon halb fünf und wir mussten ja noch in den \FOREIGN{Circus Circus} Freizeitpark.
Die zwei Achterbahnen \FOREIGN{El Loco} und der \FOREIGN{Canyon Blaster} lachten uns an.
Da der \FOREIGN{El Loco} als so \glqq brutal\grqq \, beworben wurde, setzten wir uns erstmal in den \FOREIGN{Canyon Blaster}.
Mit zwei Loopings, Schrauben und steilerem Gefälle war der aber deutlich brutaler als der luschige \FOREIGN{El Loco}.

Anschließend sind wir noch bei leichtem Nieselregen in den hoteleigenen \FOREIGN{Whirl Pool} und wurden prompt von der \FOREIGN{Security} darauf hingewiesen, das der Pool geschlossen ist.

Am Abend regnete es dann richtig und so mussten wir im Hotel auf Nahrungssuche gehen.
Das Steakhouse klang erstmal interessant, verjagte uns aber mit seinen Abwehrangeboten.
40~\$ für Vorspeisen\dots
So gingen wir zum Buffet und aßen natürlich viel zu viel.

Anschließend waren wir noch in der hoteleigenen Karaoke-Bar in der um 22$^{30}$~Uhr noch ein Kindergeburtstag gefeiert wurde.
Die zwei Kinder waren von dem Ganzen nicht so begeistert wie ihre Erziehungsberechtigten.
Undankbare Bratzen.
