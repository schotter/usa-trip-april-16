Am Abend vor der Abreise habe ich glücklicherweise noch gesteckt bekommen, dass bei Flügen aus der EU heraus der Check-In bereits drei Stunden vor Abflug erfolgen sollte.
Den ESTA-Antrag habe ich \textendash im Gegensatz zu anderen \textquotedblleft Mit\-ein\-checkern\textquotedblright \textendash auf der richtigen Interneseite gestellt, konnte in den Sicherheitsbereich und letztlich boarden.
Nach stellenweise ruppigem Flug kam der Vogel in einem Stück in Atlanta herunter.\\

Das Abenteuer Grenzkontrolle begann.
Im Flugzeug füllt man einen blauen Zettel aus auf dem man angibt wer man ist, ob man zu viel Geld dabe hat, man terroristischen Aktivitäten nachgehen möchte und am wichtigsten, wo man nächtigen wird.
Der Christian hatte mir die Adresse des Hotels noch nicht geschickt, also blieb das Feld leer.
Für eine Nation mit einem paranoiden Sicherheitsbedürfnis ist das jedoch ein absolutes No-Go.
Die Passkontrolle (Scan von allen Fingern + Foto) war dadurch schon etwas zäh.
Bei der nächsten Station gab es dafür dann einen \FOREIGN{blue folder}\footnote{blauen Schnellhefter}, was eine genaue Inspektion zur Folge hatte.
Gegraptscht wurde nicht, aber alle Fächer meines Rucksacks wurden geöffnet und meine Wäsche auf ihre Reinheit untersucht.
Im Gespräch vergewisserte man sich nochmals, dass ich keiner der Allah Uh Akbar Jungs bin und dann ging es zum normalen Security-Check weiter.
Taschen leeren, Jacke, Gürtel etc. \textbf{und} Schuhe aufs Band zum Röntgen, für einen selbst ging es zum Nacktscannen.
Der Leser stelle sich die Duftnoten von gut 40 Füßen vor\dots\\

Bis Salt Lake City gab es dann kein Theater mehr, erst im Rodeway Inn, unserem Hotel für die Nacht, gab es im Nebenzimmer ein nächtliches, das Christian komplett verschlief, ich bekam jedoch die zweiminütige ah-ah-Stimmübung mit.