Von San Diego kommt man mit der Straßenbahn bis zur mexikanischen Grenze und da wollten wir hin.
Doch davor stand erst der Gang zur Post an, denn der Christian musste noch etwas verschicken.
Es ging um Post ins Ausland, die fristgerecht ankommen musste.
Das trieb das Porto deutlich in die Höhe.
Aus ein paar Dollar wurden so 60 und mehr.
Nachdem sich der Christian für die Zustellung in 3-5 Tagen entschieden und das entsprechende Formular ausgefüllt hatte, wies ihn der Postbeamte Jesus darauf hin, dass er ein anderes Formular braucht.
Also das Ganze nochmal von vorne, wir kamen uns vor wie Asterix \& Obelix (bei der Götterprüfung im ??? TODO).

Eine gute Stunde später ging es dann zum Trolley (S-Bahn) weiter.
Ein Ticket hatten wir zwar beide gelöst, aber ob das auch gleich entwertet wurde, haben wir nicht begriffen.
Die Fahrt über waren wir daher leicht angespannt.
Daran änderten auch das komische Gebaren mancher mitfahrenden Frau nichts.
Eine sang leise, aber hörbar, vor sich hin.
Die nächste parfümierte sich und damit die halbe S-Bahn gleich mit und die letzte parfümierte ihre Handgelenke und drückte diese sogleich auf ihre Haare.

Der Grenzübertritt nach Mexiko war dann chillig.
Die Grenzpolizisten nahmen einem sogar die Arbeit ab, was das Ausfüllen der Unterlagen betraf.
Gleich nach Übertritt war es dann vorbei mit der chilligen Atmosphäre.
Alle 5~m hat uns ein Taxifahrer nach dem anderen angehupt, weil er uns in die Innenstadt fahren wollte.
Ich wollte aber laufen, was wir auch taten.
Den Weg wussten wir nicht, die am Straßenrand stehenden Taxifahrer waren aber auskunftsfreudig.

Vorbei an Werbeplakaten für schöne Zähne ging es über eine Brücke in die Innenstadt.
Das hektische Treiben und vor allem die schon wieder leicht anderen Verkehrsregeln ließen unseren Adrenalinspiegel gut ansteigen.
Unterm Strich sind wir die Straße einmal hoch und wieder runter gelaufen.
Um dann letztlich in der nicht sonderlich mexikanischen Touri-Bude amerikanische Preise zu zahlen.

Für uns war das dann erstmal genug Kulturschock und nachdem uns die Rezeptionistin vom Hostel gestern kräftig Angst gemacht hatte, wollten wir auf jeden Fall deutlich bevor Einsetzen der Dunkelheit wieder zurück sein.
Am Grenzübergang standen wir dann eine gute Stunde an, unterhielten uns mit einem Mexikaner, der schon seit längerem in den USA wohnt und zum Kauf von "Medikamenten" öfters in seine Heimat reist.
Er empfahl uns ein paar Flecken in L.A. unserem nächsten Ziel.
