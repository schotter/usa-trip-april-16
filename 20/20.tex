Gestern wollte ich mir im Wallmart noch eben zwei Einwegrasierer mitnehmen.
So wie man sie auch in Deutschland von der Tanke kennt.
Die kleinste Packungsgröße war 12.
Für 1,88~\$.

Was wir an Zeit für die USS Midway zu wenig, haben wir für das Monterey Bay Aquarium zu viel vorgesehen.
Der Eintrittspreis von 40~\$ pro Person hatte sicherlich seinen Beitrag.
Es war so langweilig, wie ich es erwartet hatte,
"Action" kam nur bei der Fütterung auf und selbst die war fad.
Denn die größeren Raubtierfische haben wenig Interesse an totem Fisch, fressen eh nur alle zwei bis drei Tage oder haben schon andere "Beifische" vernascht.
Neben den Fischen gab es noch Otter, Vögel und selbstverständlich ein Restaurant mit Gift Shops.
Die ganze Einrichtung soll dem Ami näher bringen, dass er nicht aller unüberlegt in sich reinfressen soll.
Finanziert wurde der Komplex über Spendengeldern von 100.000~\$ aufwärts.
Bei einem Betrag von 100.000~\$ könne man meinen dem Spender wird ein Flügel in der Attraktion gewidmet, aber ein Großspender wie Packard (HP) mit 50 Millionen sorgt dann eher dafür, dass einem für seine sechsstellige Summe gerade mal eine Klobrille gewidmet wird.
Aber es ist nicht so, dass ich überhaupt nichts lernen konnte.
Bei Ebbe umgreift Patrick Star eine Muschel, öffnet sie gewaltsam und frisst ihren Inhalt.

Die Fahrt nach San Francisco auf der Route~1 führte wieder an vielen Obst Plantagen vorbei auf denen sich billige Arbeitskräfte aus dem -Osten, Polen,- Süden, Mexiko, verdingen.
Wir sind dann bei ausklingendem Arbeitsverkehr am ersten Hostel angekommen und dort wäre eine Übernachtung in einem 22er Zimmer möglich gewesen, was der Prinzessin Christian nicht taugte.
Also sind wir weitergefahren zum Hostel Marin Headlands bei Sausalitos.
Dazu überquerten wir die Golden Gate Bridge, was 7.25~\$ pro Überfahrt kostet.
Abgerechnet wird das übers Kennzeichen.
Mal schauen, ob da noch was kommt.
Statt Großstadtdschungel hatten wir für die Nacht dann Natur pur.