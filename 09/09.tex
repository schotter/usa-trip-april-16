Am nächsten Morgen ging es zeitig los, 300~Meilen zum Nationalpark Zion standen auf dem Programm.
Zu Beginn sah die Gegend wie rot angemalte Alpen aus, wobei sich die Flora (sehr sehr viele Kiefern) schon deutlich abhob.
Die Felsen in Zion sind in ihrer Art dann aber so andersartig, eher einzigartig, dass man aus dem Staunen fast nicht mehr herauskommt.
Geplant war es den \FOREIGN{Angels Landing} zu wandern, was das Wetter, es goss in Strömen, nicht zuließ.

Wir fuhren daher weiter bis nach \TOWN{Page}, machten unterwegs am \TOWN{Lake Powell} halt und verweilten etwas am beeindruckenden \TOWN{Glen Canyon} Staudamm.
Von dort ging es wieder einige Meilen (240~Meilen) weiter zum Grand Canyon Nationalpark, genauer zum \FOREIGN{South Rim}, also dem südlichen Rand.

Gesehen haben wir an diesem Tag nicht mehr viel, weil es schon gut dunkel wurde und da wir noch eine Bleibe für die Nacht brauchten, war die Hotelsuche angesagt.
Am verlockensten wirkte das Canyon Inn, weil es von außen am heruntergekommensten aussah, entsprechende Hoffnungen machten wir uns auf einen guten Preis.
Die wurden an der Rezeption jedoch mit 169~\$ die Nacht zerschlagen.
Das nächste war schon etwas \glqq günstiger \grqq\, und für 143~\$ blieben wir.
Die nette Empfangsdame hatte uns eine Generalzugangskarte ausgestellt, mit der wir erstmal ins falsche, belegte Zimmer gestürmt sind.
Nachdem die Nachbarzimmer auch betretbar waren, ging es zurück in die Lobby für neue Zugangskarten.
